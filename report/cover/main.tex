\documentclass[letterpaper,12pt,oneside]{book}

% Tipografiado y preferencias regionales
\usepackage[utf8]{inputenc}
\usepackage[spanish,es-nodecimaldot,es-noindentfirst,es-tabla]{babel}
\usepackage[T1]{fontenc}
\usepackage{mathptmx}
\usepackage{csquotes,textcomp,microtype}
\decimalpoint
\unaccentedoperators

% Formato
\usepackage[top=1in, left=0.9in, right=1.25in, bottom=1in]{geometry}
\usepackage{tocloft}
\usepackage{booktabs}
\usepackage{setspace}
\usepackage{bachelorstitlepageUNAM}


% Imágenes
\usepackage{graphicx}
\usepackage{tikz} 
\graphicspath{{./figs/}}

% Bibliografía
%\usepackage[backend=biber, style=ieee]{biblatex}
%\addbibresource{references.bib}

% Matemáticas
\usepackage{amsmath}
\usepackage{amsfonts}
\usepackage{amssymb}
\usepackage{amsthm}
%\usepackage[libertine]{newtxmath}
\usepackage{fouriernc}
\allowdisplaybreaks

\newtheorem{theorem}{Teorema}[chapter]
\newtheorem{corollary}[theorem]{Corolario}
\newtheorem{lemma}[theorem]{Lema}
\newtheorem{proposition}[theorem]{Proposición}
\newtheorem{notation}[theorem]{Notación}
\newtheorem{definition}[theorem]{Definición}
\newtheorem{example}{Ejemplo}[chapter]
\newtheorem{remark}[theorem]{Observación}

\renewcommand\cftsecpresnum{\S}
\renewcommand\cftsubsecpresnum{\S}   
\renewcommand{\bibname}{Referencias y bibliografía}
\newcommand{\R}{\mathbb{R}}

% Enlaces
\usepackage[pdftex]{hyperref}
\usepackage[spanish]{cleveref}

\usepackage{lipsum}


\begin{document}
%------------------------------

    \begin{titlepage}
        \thispagestyle{empty}
        \begin{minipage}[c][0.17\textheight][c]{0.25\textwidth}
            \begin{center}
                \includegraphics[width=3.5cm, height=3.5cm]{Escudo-UNAM.pdf}
            \end{center}
        \end{minipage}
        \begin{minipage}[c][0.195\textheight][t]{0.75\textwidth}
            \begin{center}
                \vspace{0.3cm}
                \textsc{\large Universidad Nacional Autónoma de México}\\[0.5cm]
                \vspace{0.3cm}
                \hrule height2.5pt
                \vspace{.2cm}
                \hrule height1pt
                \vspace{.8cm}
                \textsc{Facultad de Ciencias}\\[0.5cm] %
            \end{center}
        \end{minipage}

        \begin{minipage}[c][0.81\textheight][t]{0.25\textwidth}
            \vspace*{5mm}
            \begin{center}
                \hskip2.0mm
                \vrule width1pt height13cm 
                \vspace{5mm}
                \hskip2pt
                \vrule width2.5pt height13cm
                \hskip2mm
                \vrule width1pt height13cm \\
                \vspace{5mm}
                \includegraphics[height=4.0cm]{Escudo-FCIENCIAS.pdf}
            \end{center}
        \end{minipage}
        \begin{minipage}[c][0.81\textheight][t]{0.75\textwidth}
            \begin{center}
                \vspace{1cm}

                {\Large\scshape Hacia el análisis de sentimientos de tuits en español de México utilizando GRUs y \textit{GloVe}}\\[.2in]

                \vspace{2cm}            

                \textsc{\LARGE P\hspace{0.4cm}R\hspace{0.4cm}O\hspace{0.4cm}Y\hspace{0.4cm}E\hspace{0.4cm}C\hspace{0.4cm}T\hspace{0.4cm}O\hspace{0.6cm}II}\\[0.5cm]
                \textsc{\large Presenta:}\\[0.5cm]
                \textsc{\large {Jorge Esteban Mendoza Ortiz}}\\[2cm]          

                \vspace{0.5cm}

                {\large\scshape Tutores:\\[0.3cm] {M. en C. María Fernanda Sánchez Puig \\ 
Sergio Miguel Fernández Martínez}}\\[.2in]

                \vspace{0.5cm}

                \large{Ciudad de México,}{ }{2022}
            \end{center}
        \end{minipage}
    \end{titlepage}

%---------------------------------
\frontmatter
\tableofcontents

\mainmatter


\section{Introducción}

\backmatter%@sglvgdor



\end{document}